\documentclass[12pt]{article}

\usepackage{amsmath}
\usepackage{amsfonts}
\usepackage{graphicx}
\usepackage[subrefformat=parens]{subcaption}
\usepackage{booktabs}
\usepackage{url}
\usepackage{hyperref}

\title{Project title}
\author{Andrew Floren}
\date{}

\begin{document}
\maketitle{}

\section{Aims}
Traditionally, fMRI has been used to study the spatial and temporal correlation of metabolic activity in the brain with various external activities or tasks.
A recent trend in the field has seen machine learning applied to fMRI data in an attempt to blindly predict the the external activity or task from the pattern of activation \cite{a,b,c}.
In this way, researchers hope to uncover more complex relationships between the patterns of activation and the external activity.
Thus far, researchers have been successful in differentiating the patterns of activation that result from subjects viewing broad categories of objects.
For example, the researchers in \cite{a} were able to differentiate the pattern of activation created by a subject viewing an image of a face from the pattern created by a subject viewing a place.
Using these predictive methods as tools, other researchers have assembled a number of insightful cognitive and behavioral studies \cite{d,e,f}.
However, these studies are limited to the dimensions that can be resolved by the current predictive methods; i.e., we can resolve the difference between a subject viewing a face and a place, but not between a subject viewing two different faces.
Our goal is to increase both the number of resolvable dimensions, as well as the resolution along those dimensions.

The current lack of resolvable dimensions is due in large part to the novelty of the approach.
There is a wealth of information regarding correlation analyses between functional activation and all manner tasks, but so far the number of tasks that predictive methods have been applied to is limited.
The resolvable resolution along dimensions is limited in part by measurement noise; this includes noise introduced by the scanner as well as cognitive processes.
However, the resolution is also limited by the quality of the predictive methods utilized.
Thus far, only relatively simple machine learning techniques have been employed due to the necessary cross-disciplinary collaboration required for more complex machine learning techniques.

To increase the number of resolvable dimensions, we will evaluate the performance of predictive methods on new tasks using the results from more traditional fMRI experiments as guidelines for finding tasks with a high probability of success.
Our own preliminary results suggest that it should be possible to predict the number of objects a subject is viewing and this is where we will be focusing the initial stages of our research. 
Additionally, we will experiment with more state of the art machine learning algorithms in order to improve the resolution with which we can measure these dimensions.

\section{Background}

\section{Preliminary/Expected Results}

\section{Research Design}

\end{document}